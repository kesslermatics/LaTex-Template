%
%  Vorlage/Template fuer #EBT
%
%  Created by Prof. Dr. Detlef Kreuz on 2010-08-14.
%  Copyright (c) 2010 . All rights reserved.
%
\documentclass[12pt,toc=bib,toc=listof]{scrreprt}
\usepackage[ngerman]{babel} 
\usepackage[utf8]{inputenc}
\usepackage[T1]{fontenc}
\usepackage{lmodern}
\usepackage{setspace}
\usepackage{hyperref}
\usepackage{fixltx2e}
\usepackage{acronym}
\usepackage[automake,toc]{glossaries}
\usepackage{float}
\usepackage{booktabs}
\usepackage{listings}
\usepackage{makecell}
\usepackage[table,xcdraw]{xcolor}
\usepackage[]{capt-of}
\usepackage[style=apa,backend=biber,natbib]{biblatex}
\usepackage{ifpdf}
\usepackage{subcaption}
\ifpdf
\usepackage[pdftex]{graphicx}
\else
\usepackage{graphicx}
\fi
\usepackage{tikz}
\usetikzlibrary{shapes}
\usepackage[headsepline]{scrlayer-scrpage}
\usepackage{pgfgantt}
\usepackage{xcolor}
\addbibresource{references.bib}

\makeglossaries


\newglossaryentry{Phytoplankton}
{
    name={Phytoplankton}, 
    description={Phytoplankton sind winzige, pflanzliche aquatische Organismen, die durch Photosynthese organische Verbindungen produzieren, die von anderen Organismen als Nahrung genutzt werden können}
}


\newcommand{\keywords}[1]{\par\noindent{\small{\textbf{Keywords}}--- #1}}

\hypersetup{
  ,colorlinks=true
  ,linkcolor=blue
  ,citecolor=blue
  ,filecolor=blue
  ,urlcolor=blue
}

  \lstnewenvironment{rcode}[1][]{
  \lstset{
    language=R,
    basicstyle=\small\ttfamily,
    columns=fullflexible,
    keywordstyle=\color{blue}\bfseries,
    commentstyle=\color{gray},
    stringstyle=\color{green!60!black},
    showstringspaces=false,
    breaklines=true,
    postbreak=\mbox{\textcolor{red}{$\hookrightarrow$}\space},
    morekeywords={},
    frame=single,
    frameround=tttt,
    backgroundcolor=\color{gray!10},
    framesep=5pt,
    xleftmargin=5pt,
    xrightmargin=5pt,
    tabsize=2,
    aboveskip=15pt,   % Abstand oberhalb des Codeblocks
    belowskip=15pt,   % Abstand unterhalb des Codeblocks
    rulecolor=\color{gray!8},  % Rahmenlinienfarbe
    frame=tb,  % Rahmenlinien an allen vier Seiten
    backgroundcolor=\color{gray!10},  % Hintergrundfarbe des Codeblocks
    fillcolor=\color{gray!10},  % Hintergrundfarbe des Codeblocks für Zeilenumbrüche
  }
}{}
  \lstnewenvironment{routput}[1][]{
  \lstset{
    language=R,
    basicstyle=\small\ttfamily,
    columns=fixed,
    keywordstyle=\color{blue}\bfseries,
    commentstyle=\color{gray},
    stringstyle=\color{green!60!black},
    showstringspaces=false,
    breaklines=true,
    postbreak=\mbox{\textcolor{red}{$\hookrightarrow$}\space},
    morekeywords={},
    frame=single,
    frameround=tttt,
    backgroundcolor=\color{gray!10},
    framesep=5pt,
    xleftmargin=5pt,
    xrightmargin=5pt,
    tabsize=7,   % Tabulator mit 7 Leerzeichen
    #1,
    aboveskip=15pt,
    belowskip=15pt,
  }
}{}

%%%%%%%%%%%%%%%%%%%%%%%%%%%%%%%%%%%%% % (fold)
\newcommand{\ebttopic}{Lorem Ipsum}
\newcommand{\ebtstudentname}{Name: Robert Kessler}
\newcommand{\ebtstudentid}{Matrikelnummer: 205015}
\newcommand{\ra}[1]{\renewcommand{\arraystretch}{#1}}
\urldef{\ebtstudentmail}{\url}{E-Mail}
%
%%%%%%%%%%%%%%%%%%%%%%%%%%%%%%%%%%%%% % (end)

\pagestyle{scrheadings}

\clearscrheadfoot
%\ihead{\ebttopic}
\ohead{\pagemark}
\renewcommand*{\chapterpagestyle}{scrheadings}
\renewcommand*{\chapterheadstartvskip}{}

\titlehead{\flushright
% \includegraphics[scale=0.1]{images/HHNLogo.jpg}
}
\subject{Side \\ Title}
\title{\ebttopic}
\author{\ebtstudentname \\ \ebtstudentid}
%% Datum nie auf einen festen Wert setzen
\publishers{\flushleft Eingereicht bei: \\Referent: Test Prof\\Korreferent: Test Prof 2}

%\pagestyle{headings}

\begin{document}
\interlinepenalty=5
\pagenumbering{roman} 
\selectlanguage{ngerman}
\maketitle
\begin{abstract}
    \begin{center}
        \large \textbf{Abstract}
        \end{center}
\noindent Lorem Ipsum dolor sit amet, consectetur adipiscing elit. Nullam nec pur
\keywords{Internet of Things (IoT), Algenblüten, Chlorophyll-a}
\end{abstract}

\tableofcontents

\include{chapters/lists/abkuerzungsverzeichnis.tex}

\printglossary[nonumberlist]

\listoffigures
\listoftables

\onehalfspacing

\newpage
\pagenumbering{arabic}

\chapter{Einleitung} % (fold)
\label{sec:Einleitung}

Der folgende Teil dient dazu, das Thema vorzustellen und einen Überblick über den Inhalt und die Ziele der Arbeit zu geben. 

% \input{chapters/einleitung/hintergrund_und_motivation.tex}


% \input{chapters/hauptteil/theoretische_grundlagen/theoretische_grundlagen.tex}







% \input{chapters/schluss/fazit/fazit.tex}

\appendix
\include{chapters/lists/literaturverzeichnis.tex}
\onehalfspace
\section*{Ehrenwörtliche Erklärung}

Hiermit versichere ich, die vorliegende Arbeit ohne fremde Hilfe und nur unter Verwendung der angegebenen Hilfsmittel selbstständig verfasst zu haben. 
Alle Stellen, die wörtlich oder sinngemäß aus veröffentlichten oder nicht veröffentlichten Arbeiten anderer entnommen sind, habe ich kenntlich gemacht.\bigskip

\noindent
Heilbronn, den \today

\vspace*{2cm}
%\noindent


\begin{tabular}{@{}l@{}}\hline
\rule{0pt}{2ex}
Robert Kessler
\end{tabular}

% \rule{0pt}{2ex} = Wert "2ex" regelt den Abstand zwischen Linie und Namen
% Länge des Namens regelt die Länge der Linie automatisch

\end{document}

\end{document}

